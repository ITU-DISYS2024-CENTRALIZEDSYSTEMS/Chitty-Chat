% report.tex

\documentclass[a4paper,11pt]{article}

% Import packages
\usepackage[a4paper]{geometry}
\usepackage[utf8]{inputenc}
\usepackage{amsmath}
\usepackage{amssymb}
\usepackage{graphicx}


% Change enumerate environments you use letters
\renewcommand{\theenumi}{\alph{enumi}}

% Set title, author name and date
\title{Chitty-Chat}
\author{Johannes Jørgensen, Kevin Skovgaard Gravesen} 
\date{\today}

\begin{document} 

\maketitle

\subsection*{Discuss, whether you are going to use server-side streaming, client-side streaming, or bidirectional streaming?}

\subsection*{Describe your system architecture - do you have a server-client architecture, peer-to-peer, or something else?}
We have implemented our Chitty-Chat program throught the server-client architecture. 
This means that a server is launched and is only responsible for the communication between different clients. This means that the clients are only responsible for
sending acceptable data from the client to the server.
\subsection*{Describe what  RPC methods are implemented, of what type, and what messages types are used for communication}
We have implemented a single RPC method called \verb|JoinConversation| that both takes and return a \textit{stream} of type \verb|Message|.
The \verb|Message| type is the message type used for communication. It contains the fields:
\begin{itemize}
    \item \verb|timestamp|: A Lamport timestamp
    \item \verb|username|: The username of the sender
    \item \verb|content|: The content of the message
\end{itemize} 
\subsection*{Describe how you have implemented the calculation of the Lamport timestamps}
\subsection*{Provide a diagram, that traces a sequence of RPC calls together with the Lamport timestamps, that corresponds to a chosen sequence of interactions: Client X joins, Client X Publishes, ..., Client X leaves. Include documentation (system logs) in your appendix.}
Both the server and the client have a \verb|Lamport| clock initilized on start, that is set to \verb|0|. This is also the case for newcomers that connect to the server. 
The \verb|Lamport| clock is handled with these conditions:\\
\textbf{Client:}
\begin{itemize}
    \item When a client sends a message, the \verb|Lamport| clock is incremented by \verb|1|.
    \item When a client receives a message, the \verb|Lamport| clock is incremented by \verb|1| if the timestamp of the received message is greater than the current \verb|Lamport| clock.
\end{itemize}
\textbf{Server:}
\begin{itemize}
    \item When the server receives a message, the \verb|Lamport| clock is incremented by \verb|1| if the timestamp of the received message is greater than the current \verb|Lamport| clock.
    \item When the server broadcasts a message, the \verb|Lamport| will be incremented by \verb|1|.
    \item If a client disconnectes from the server, \verb|Lamport| clock is incremented by \verb|1| - Then the server will broadcast the disconnection.
    \item If a client connects to the server, \verb|Lamport| clock is incremented by \verb|1|- Then the server will broadcast the disconnection.
\end{itemize}
\subsection*{Provide a diagram, that traces a sequence of RPC calls together with the Lamport timestamps}
\includegraphics[width=\textwidth]{chat.png}
Server   - Starting
Client 1 - Joined the conversation
Client 2 - Joined the conversation
Client 1 - Sends a message
Client 2 - Leaves the conversation

\subsection*{Provide a link to a Git repo with your source code in the report}
\subsection*{Include system logs, that document the requirements are met, in the appendix of your report}
\subsection*{Include a readme.md file that describes how to run your program.}

\end{document}